%!TEX root = article.tex


\section{Convergence analysis}\label{sec:analysis}

We give four convergence results, in increasing order of interest: (i) a stationary distribution
convergence property for the random Sinkhorn algorithm ; (ii) a global
convergence property for the online Sinkhorn algorithm without noise ; (iii) an
approximate convergence property for the online Sinkhorn algorithm with fixed
batch-size and (iv) an exact global convergence result for online Sinkhorn
with increasing batch sizes.

We make the following classic assumptions on the cost regularity and
distribution compactness, which is necessary to obtain a uniform
soft $C$-transform contractance ratio.

\begin{assumption}\label{ass:lip}
    The cost $C: \Xx \times \Xx \to \RR$ is $L$-lipschitz, and $\Xx$ is  compact.
\end{assumption}

\subsection{Randomized Sinkhorn}

We first state a result concerning the randomized Sinkhorn algorithm~\eqref{eq:updates}, which corresponds to
\autoref{alg:online_sinkhorn} with step-size $\eta_t = 1$.

\begin{proposition}\label{prop:markov}
    Under \autoref{ass:lip}, the randomized Sinkhorn algorithm \eqref{eq:updates} yields a time-homogeneous
    Markov chain ${(f_t, g_t)}_t$ which is $(\hat \alpha_s, \hat \beta_s)_{s \leq
    t}$ measurable, and converges in law towards a stationary distribution
    $(F_\infty, G_\infty) \in \Pp(\Cc(\Xx)^2)$ independent of the initialization
    point $(f_0, g_0)$.
\end{proposition}

This result follows from \citet{diaconis_iterated} convergence theorem on
iterated random functions which are contracting on average. We simply use the
fact that $\Ctrans{\cdot}{\hat \beta}$ and $\Ctrans{\cdot}{\hat \alpha}$ are
\textit{always} contracting, independently of the distributions $\hat \alpha$ and
$\hat \beta$, for the variational norm $\Vert \cdot \Vert_{\text{var}}$.

\paragraph{Out-of-loop averages.} Note that using the law of large number for Markov chains
\citep{breiman_strong_1960}, the out-of-loop averages $\exp(-\bar f_t)$
converge to $\EE[\exp(-F_\infty)] \in \Cc(\Xx)$ for $\gamma_t = \frac{1}{t}$. This expectation verifies the following fixed point equations
\begin{align}
    \EE[\exp(-F_\infty)] &=
     \dotp{\beta}{\EE[\exp(G_\infty)] \exp(-C)}, \\
    \EE[\exp(-G_\infty)] &=
     \dotp{\alpha}{\EE[\exp(F_\infty)] \exp(-C)}.
\end{align}
These fixed point equations are close to the Sinkhorn fixed point equations, and
get closer as $\varepsilon$ increases, since $\varepsilon \EE[\exp(\pm F_\infty /
\varepsilon)] \to \EE[F_\infty]$ as $\varepsilon \to \infty$. Running the random
Sinkhorn algorithm with out-of-loop averaging fails to provide exactly the dual solution.
However, it defines an approximate solution of the original problem whose accuracy depends on $\epsilon$. 
%
We leave the quantification of this approximation for future work.

%%%
\subsection{Noise-free online Sinkhorn.}

Variance reduction is therefore necessary, to ensure that the limit stationary
distribution is deterministic. The following proposition shows that the modified "slowed-down" online Sinkhorn algorithm converges in the absence of noise.

\begin{proposition}\label{eq:deterministic}
    We suppose that $\hat \alpha_t = \alpha$, $\hat \beta_t = \beta$ for all
    $t$. Then the updates \eqref{eq:updates} yields a (deterministic) sequence $(f_t, g_t)_t$ such
    that 
    \begin{gather}
        \Vert f_t - f^\star \Vert_{\text{var}} 
        + \Vert g_t - g^\star \Vert_{\text{var}} \to 0,\quad\text{and} \\
        \frac{1}{2} \dotp{\alpha}{f_t + \Ctrans{g_t}{\alpha}} + \dotp{\beta}{g_t + \Ctrans{f_t}{\beta}} 
         \to \Ww(\alpha, \beta).
    \end{gather}
\end{proposition}
Note that, due to the fact that we perform simultaneous updates, we only obtain
the convergence of $f_t \to f^\star + A$, and $g_t \to g^\star$, where $f^\star$
and $g^\star$ are solutions of \eqref{eq:wass} and $A$ is a constant depending
on initialization. This is only a small caveat, as we can average the potentials
and their soft $C$-transform as in \eqref{eq-dist-est} to remove the offset $A$.
This is not necessary when using the Sinkhorn distance as a loss for training
purposes, e.g. for generative modeling or barycenter estimation as in
\citet{staib2017parallel}. Backpropagation through the Sinkhorn distance indeed
relies only on the gradients of the potentials $\nabla_x f^\star(\cdot)$,
$\nabla_y g^\star(\cdot)$ \citep[e.g.][]{cuturi2018semidual}.
%%%
\subsection{Online Sinkhorn}

We make the following \citet{robbins1951stochastic} assumption on the weight sequence

\begin{assumption}\label{ass:weights}
    The weight sequence ${(\eta_t)}_t$ is square summable but not summable: 
    $\sum \eta_t = \infty$ and $\sum \eta_t^2 < \infty$.
\end{assumption}


We first state an approximate convergence result for the online Sinkhorn algorithm with fixed batch-size $n(t) = n$.

\begin{proposition}\label{prop:convergence_bis}
    We assume \autoref{ass:lip} and \ref{ass:weights}. The online Sinkhorn algorithm (\autoref{alg:online_sinkhorn}) yields a sequence $(f_t, g_t)$ that reaches a
    ball centered around $f^\star, g^\star$ for the variational norm $\Vert
    \cdot \Vert_{\var}$.
     Namely, there exists $T > 0$, $C > 0$ such that for all $t > T$, almost surely
    \begin{equation}
        \Vert f_t - f^\star \Vert_{\var}
        + \Vert g_t - g^\star \Vert_{\var} 
        \leq \frac{C}{\sqrt{n}} 
    \end{equation}
\end{proposition}
We cannot state an exact convergence result in this case because the noise due
to replacing $\alpha$ and $\beta$ with $\hat \alpha$ and $\hat \beta$ in the
$C$-transform does not decrease with time --- slowing down the Sinkhorn
iterations is not enough at we must ensure that $\sum \eta_t = \infty$. On the
other hand, slowing-down iterations allows to obtain an approximation in
$\Oo(\frac{1}{\sqrt{n}}$). This is comparable to the size of the error made by
performing Sinkhorn without resampling \citep{2019-Genevay-aistats}.  We show in
the experiment section $(\autoref{sec:exps})$ that online Sinkhorn strongly
outperforms the regular Sinkhorn algorithm thanks to repeated sampling. This
suggests that the constants appearing in the bounds of online Sinkhorn are much
better than the ones appearing in the sample complexity of the regular Sinkhorn
algorithm.

Finally, we show the almost sure convergence of the online Sinkhorn algorithm
with slightly increasing batch-size $n(t)$, as specified in the following
assumption.

\begin{assumption}\label{ass:double_weights}
    For all $t > 0$, the batch-size $n(t) = \frac{n}{w_t^2}$ is an integer. ${(\eta_t)}_t$ is
    not summable and ${(w_t \eta_t)}_t$ is summable: $\sum w_t \eta_t <
    \infty$ and $\sum \eta_t = \infty$.
\end{assumption}
We then have the following global convergence result.

\begin{proposition}\label{prop:convergence_true}
    We assume \autoref{ass:lip}, and
    \ref{ass:double_weights}. Almost surely, the iterates of online Sinkhorn
    (\autoref{alg:online_sinkhorn}) converge, and we have
    \begin{equation}
        \Vert f_t - f^\star \Vert_{\var} + \Vert g_t - g^\star \Vert_{\var} \to 0.
    \end{equation}
\end{proposition}

\paragraph{Choosing batch-sizes.}

Online Sinkhorn thus works for $\eta_t = \frac{1}{t^a}$ with $a \in [0, 1]$,
provided that we use batch-sizes of size $n(t) = n\, t^{2(1-a)} \log^c t$, with $c >
1$. Slowing down Sinkhorn iterations thus permits to work with batches whose
size increases more slowly. The limit case when $a = 1$ requires only $b > 0$
can accommodate batch-sizes growing arbitrarily slowly. 

Note that using growing batch-size simply tells us that we should increase the
computational budget of single iterations over time: the overall complexity
after seeing $n_t$ samples remain in $\Oo(n_t^2)$. We use fixed batch-sizes in
experiments as those already perform well.