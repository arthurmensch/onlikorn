\documentclass[a4paper, 10pt]{article}
%!TEX root = article.tex
\usepackage[utf8]{inputenc}
\usepackage[T1]{fontenc}

% Bold math
\usepackage{bm}
\usepackage{dsfont}
\usepackage{csquotes}

% Micro-typing
\usepackage{xspace}
\usepackage{nicefrac}
\usepackage[super]{nth}
\usepackage{microtype}

\usepackage{multibib}

\usepackage[english]{babel}

% Compact itemize
\usepackage{enumitem}

% Tables
\usepackage{booktabs}
\usepackage{makecell}


\usepackage{graphicx}
\graphicspath{{./figures/}}

\usepackage{subcaption}

% For colors
\usepackage{xcolor}

% Algorithm
\usepackage{algorithm}
\usepackage[noend]{algpseudocode}
\algrenewcommand{\algorithmiccomment}[1]{$\vartriangleright$ #1}
\algrenewcommand{\algorithmicreturn}{\textbf{Return: }}
\algnewcommand\algorithmicinput{\textbf{Input: }}
\algnewcommand\Input{\State \algorithmicinput}

% Hyper reference
\usepackage[colorlinks=true,bookmarks=true,linkcolor=blue,
urlcolor=blue,citecolor=blue,breaklinks=true]{hyperref}

\usepackage{url}

\makeatletter
\addto\extrasenglish{%
\renewcommand{\appendixautorefname}{App.}%
\renewcommand{\figureautorefname}{Fig.}%
\renewcommand{\tableautorefname}{Table}%
\renewcommand{\sectionautorefname}{\S\@gobble}%
\renewcommand{\subsectionautorefname}{\S\@gobble}%
\renewcommand{\subsubsectionautorefname}{\S\@gobble}%
}
\makeatother

% Float parameters, for more full pages.
\renewcommand{\topfraction}{0.9}	% max fraction of floats at top
\renewcommand{\bottomfraction}{0.8}	% max fraction of floats at bottom
\renewcommand{\textfraction}{0.07}	% allow minimal text w. figs
%   Parameters for FLOAT pages (not text pages):
\renewcommand{\floatpagefraction}{0.6}	% require fuller float pages
%    % N.B.: floatpagefraction MUST be less than topfraction !!
\renewcommand{\dbltopfraction}{.95}  % double page floats at the top
\renewcommand{\dblfloatpagefraction}{.6}
\setcounter{totalnumber}{4}
\setcounter{topnumber}{3}
\setcounter{dbltopnumber}{3}
\setcounter{bottomnumber}{2}

% Wide pages for supplementary figures
\usepackage{changepage}                 % adjust margins for selected portions
% wide page for side by side figures, tables, etc
\newlength{\offsetpage}
\setlength{\offsetpage}{3.0cm}
\newenvironment{widepage}{\begin{adjustwidth}{-\offsetpage}{-\offsetpage}%
    \addtolength{\textwidth}{2\offsetpage}}%
{\end{adjustwidth}}


% MATHS (AMS)
\usepackage{amsmath}
\usepackage{amsfonts} 
\usepackage{amssymb}
\usepackage{mathrsfs}
\usepackage{mathtools} % for psmallmatrix
\mathtoolsset{showonlyrefs}

\usepackage{amsthm}

\newtheorem{theorem}{Theorem}
\newtheorem{proposition}{Proposition}
\newtheorem{definition}{Definition}
\newtheorem{corollary}{Corollary}
\newtheorem{lemma}{Lemma}
\newtheorem{remark}{Remark}
\newtheorem{example}{Example}
\newtheorem{assumption}{Assumption}

\newtheorem{subassumptioninner}{Assumption}
\newenvironment{subassumption}[1]{%
  \renewcommand\thesubassumptioninner{#1}%
  \subassumptioninner
}{\endsubassumptioninner}

\newcommand{\definitionautorefname}{Def.}%
\newcommand{\propositionautorefname}{Prop.}%
\newcommand{\corollaryautorefname}{Corollary}%
\newcommand{\assumptionautorefname}{Ass.}%
\newcommand{\subassumptioninnerautorefname}{Ass.}%
\newcommand{\algorithmautorefname}{Alg.}%
\newcommand{\lemmaautorefname}{Lemma}
\newcommand{\remarkautorefname}{Rmk.}

\newcommand{\todo}[1]{{\color{red} TODO: #1}}

%!TEX root = notes.tex
%% Symboles avec double lignes
\newcommand{\NN}{\mathbb{N}}
\newcommand{\CC}{\mathbb{C}}
\newcommand{\GG}{\mathbb{G}}
\newcommand{\LL}{\mathbb{L}}
\newcommand{\PP}{\mathbb{P}}
\renewcommand{\SS}{\mathbb{S}}
\newcommand{\QQ}{\mathbb{Q}}
\newcommand{\RR}{\mathbb{R}}
\newcommand{\VV}{\mathbb{V}}
\newcommand{\ZZ}{\mathbb{Z}}
\newcommand{\FF}{\mathbb{F}}
\newcommand{\KK}{\mathbb{K}}
\newcommand{\TT}{\mathbb{T}}
\newcommand{\UU}{\mathbb{U}}
\newcommand{\EE}{\mathbb{E}}
\newcommand{\XX}{\mathbb{X}}
\newcommand{\YY}{\mathbb{Y}}
%% Symboles arrondis
\newcommand{\Aa}{\mathcal{A}}
\newcommand{\Bb}{\mathcal{B}}
\newcommand{\Cc}{\mathcal{C}}
\newcommand{\Dd}{\mathcal{D}}
\newcommand{\Ee}{\mathcal{E}}
\newcommand{\Ff}{\mathcal{F}}
\newcommand{\Gg}{\mathcal{G}}
\newcommand{\Hh}{\mathcal{H}}
\newcommand{\Ii}{\mathcal{I}}
\newcommand{\Jj}{\mathcal{J}}
\newcommand{\Kk}{\mathcal{K}}
\newcommand{\Ll}{\mathcal{L}}
\newcommand{\Mm}{\mathcal{M}}
\newcommand{\Nn}{\mathcal{N}}
\newcommand{\Oo}{\mathcal{O}}
\newcommand{\Pp}{\mathcal{P}}
\newcommand{\Qq}{\mathcal{Q}}
\newcommand{\Rr}{\mathcal{R}}
\newcommand{\Ss}{\mathcal{S}}
\newcommand{\Tt}{\mathcal{T}}
\newcommand{\Uu}{\mathcal{U}}
\newcommand{\Vv}{\mathcal{V}}
\newcommand{\Ww}{\mathcal{W}}
\newcommand{\Xx}{\mathcal{X}}
\newcommand{\Yy}{\mathcal{Y}}
\newcommand{\Zz}{\mathcal{Z}}
\newcommand{\KL}{\text{KL}}

% matrices
\newcommand{\A}{\bm{A}}
\newcommand{\B}{\bm{B}}
\newcommand{\C}{\bm{C}}
\newcommand{\D}{\bm{D}}
\newcommand{\E}{\bm{E}}
\newcommand{\F}{\bm{F}}
\newcommand{\G}{\bm{G}}
\newcommand{\Hz}{\bm{H}}
\newcommand{\I}{\bm{I}}
\newcommand{\J}{\bm{J}}
\newcommand{\K}{\bm{K}}
\newcommand{\Lz}{\bm{L}}
\newcommand{\M}{\bm{M}}
\newcommand{\N}{\bm{N}}
\newcommand{\Oz}{\bm{O}}
\newcommand{\Pz}{\bm{P}}
\newcommand{\Q}{\bm{Q}}
\newcommand{\R}{\bm{R}}
\newcommand{\Sz}{\bm{S}}
\newcommand{\T}{\bm{T}}
\newcommand{\U}{\bm{U}}
\newcommand{\V}{\bm{V}}
\newcommand{\W}{\bm{W}}
\newcommand{\X}{\bm{X}}
\newcommand{\Y}{\bm{Y}}
\newcommand{\Z}{\bm{Z}}

% Vectors
\renewcommand{\a}{\bm{a}}
\renewcommand{\b}{\bm{b}}
\renewcommand{\c}{\bm{c}}
\renewcommand{\d}{\textrm{d}}
\newcommand{\e}{\bm{e}}
\newcommand{\f}{\bm{f}}
\newcommand{\g}{\bm{g}}
\newcommand{\h}{\bm{h}}
\renewcommand{\i}{\bm{i}}
\renewcommand{\j}{\bm{j}}
\renewcommand{\k}{\bm{k}}
\renewcommand{\l}{\bm{l}}
\newcommand{\m}{\bm{m}}
\newcommand{\n}{\bm{n}}
\renewcommand{\o}{\bm{o}}
\newcommand{\p}{\bm{p}}
\newcommand{\q}{\bm{q}}
\renewcommand{\r}{\bm{r}}
\newcommand{\s}{\bm{s}}
\renewcommand{\t}{\bm{t}}
\renewcommand{\u}{\bm{u}}
\renewcommand{\v}{\bm{v}}
\newcommand{\w}{\bm{w}}
\newcommand{\x}{\bm{x}}
\newcommand{\y}{\bm{y}}
\newcommand{\z}{\bm{z}}

% greek
\newcommand{\al}{\alpha}
\newcommand{\la}{\lambda}
\newcommand{\ga}{\gamma}
\newcommand{\Ga}{\Gamma}
\newcommand{\La}{\Lambda}
\newcommand{\si}{\sigma}
%\newcommand{\Si}{\Sigma}
\newcommand{\be}{\beta}
\newcommand{\de}{\delta}
\newcommand{\De}{\Delta}
\renewcommand{\phi}{\varphi}
\renewcommand{\th}{\theta}
\newcommand{\om}{\omega}
\newcommand{\Om}{\Omega}

% hat, tilde
\newcommand{\hf}{\hat f}
\newcommand{\wtf}{\tilde f}
\newcommand{\tx}{\tilde x}
\newcommand{\ta}{\tilde a}
\newcommand{\tb}{\tilde b}
\newcommand{\ty}{\tilde y}
\newcommand{\tu}{\tilde u}
\newcommand{\tv}{\tilde v}
\newcommand{\tga}{\tilde \ga}
\newcommand{\tf}{\wt{f}}



%%%%%%%%%%%%%%% MATHS OPERATORS %%%%%%%%%%%%%%%%%
\newcommand{\ins}[1]{\mathrm{#1}}

\DeclareMathOperator{\realp}{\Rr e}
\DeclareMathOperator{\imagp}{\Ii m}
\DeclareMathOperator{\Ker}{Ker}
\DeclareMathOperator{\Hom}{Hom}
\DeclareMathOperator{\End}{End}
\DeclareMathOperator{\tr}{tr}
\DeclareMathOperator{\Tr}{Tr}
\DeclareMathOperator{\Supp}{Supp}
\DeclareMathOperator{\Sign}{Sign}
\let\Im\relax
\DeclareMathOperator{\Im}{Im}
\DeclareMathOperator{\Corr}{Corr}
\DeclareMathOperator{\sign}{sign}
\DeclareMathOperator{\supp}{supp}

\DeclareMathOperator{\cas}{cas}
\DeclareMathOperator{\sinc}{sinc}
\DeclareMathOperator{\cotan}{cotan}
\DeclareMathOperator{\Card}{Card}
\DeclareMathOperator{\GCD}{GCD}
\DeclareMathOperator{\grad}{grad}
\DeclareMathOperator{\Diag}{Diag}
\DeclareMathOperator{\rank}{rank}
\DeclareMathOperator{\conv}{conv}
\DeclareMathOperator{\interop}{int}

\newcommand{\ps}[2]{\langle #1,#2\rangle}

% regularity
\newcommand{\Calpha}{\mathrm{C}^\al}
\newcommand{\Cbeta}{\mathrm{C}^\be}
\newcommand{\Cal}{\text{C}^\al}
\newcommand{\Ctwo}{\text{C}^{2}}
\newcommand{\Calt}[1]{\text{C}^{#1}}
\newcommand{\Cder}[1]{\mathscr{C}^{#1}}
% Lp spaces
\newcommand{\lone}{\ell^1}
\newcommand{\ltwo}{\ell^2}
\newcommand{\linf}{\ell^\infty}
\newcommand{\Lone}{\text{\upshape L}^1}
\newcommand{\Ltwo}{\text{\upshape L}^2}
\newcommand{\Linf}{\text{\upshape L}^\infty}
\newcommand{\lzero}{\ell^0}
\newcommand{\lp}{\ell^p}
% circle
\newcommand{\Sun}{\text{S}^1}
% little space after forall
\newcommand{\foralls}{\forall \,}
%% for derivatives
\newcommand{\dd}{\ins{d}}
%% Use french comparaison operator
\renewcommand{\leq}{\leqslant}
\renewcommand{\geq}{\geqslant}


%%%%%%%%%%%%%%% MATHS CONSTRUCTS %%%%%%%%%%%%%%%

%% over-symbols
\newcommand{\ol}[1]{\overline{#1}}
\newcommand{\wh}[1]{\widehat{#1}}
\newcommand{\whwh}[1]{\hat{\hat{#1}}}
\newcommand{\wt}[1]{\widetilde{#1}}
%% partial derivatives
\newcommand{\pd}[2]{ \frac{ \partial #1}{\partial #2} }
\newcommand{\pdd}[2]{ \frac{ \partial^2 #1}{\partial #2^2} }
%% nice epsilon
\renewcommand{\epsilon}{\varepsilon}
%% Pour avoir un joli i pour les complexes
\renewcommand{\imath}{\mathrm{i}}
\newcommand{\interior}[1]{\ensuremath{\overset{\circ}{#1}}}

%% Legendre symbol
\newcommand{\legsymb}[2]{ \genfrac{(}{)}{}{}{#1}{#2} }
%% exposant pour les ordinaux
\newcommand{\ordin}[2]{ ${#1}^{ \text{#2} }$ }
%% Dot product and cross product
% \newcommand{\dotp}[2]{ \left\langle #1,\,#2 \right\rangle }
\newcommand{\dotp}[2]{\langle #1,\,#2\rangle}
\newcommand{\dotps}[2]{\langle #1,\,#2\rangle}
\newcommand{\crossp}[2]{ #1 \hat #2 }
\newcommand{\seg}[2]{\llbracket #1,\,#2 \rrbracket}
\newcommand{\brac}[1]{\left[#1\right]}
%\newcommand{\norm}[1]{|\!| #1 |\!|}
\newcommand{\norm}[1]{\left\| #1 \right\|}
\newcommand{\snorm}[1]{\| #1 \|}
\newcommand{\normb}[1]{\Big|\!\Big| #1 \Big |\!\Big|}
\newcommand{\normB}[1]{\left|\!\left| #1 \right|\!\right|}
\DeclareMathOperator{\diverg}{div}
\DeclareMathOperator{\Prox}{Prox}
\newcommand{\normT}[1]{\norm{#1}_{\text{T}}}
\newcommand{\normu}[1]{\norm{#1}_{1}}
\newcommand{\normi}[1]{\norm{#1}_{\infty}}
\newcommand{\normd}[1]{\norm{#1}_{2}}
\newcommand{\normz}[1]{\norm{#1}_{0}}
\newcommand{\abs}[1]{\left\lvert#1\right\rvert} % modified by Vincent
\newcommand{\absb}[1]{\Big| #1 \Big|}

%% Function definition
\newcommand{\func}[4]{ {\left\{  \begin{array}{ccc} #1 & \longrightarrow & #2 \\ #3 & \longmapsto & #4 \end{array}  \right.} }
% transpose
\newcommand{\transp}[1]{ {#1}^{\ins{T}} }
% l'identit�
\newcommand{\Id}{\ins{Id}}
% egal par d�finition
\newcommand{\eqdef}{\triangleq}

\DeclareMathOperator*{\argmin}{argmin}
\DeclareMathOperator*{\argmax}{argmax}
% \DeclareMathOperator*{\sup}{sup}

%% parenthesis
\newcommand{\pa}[1]{\left( #1 \right)}
\newcommand{\bpa}[1]{\big( #1 \big)}
\newcommand{\choice}[1]{ %
	\left\{ %
		\begin{array}{l} #1 \end{array} %
	\right. }
% ensembles
\newcommand{\set}[1]{ \{ #1 \} }
\newcommand{\condset}[2]{ \left\{ #1 \;;\; #2 \right\} }


%%%%%%%%%%%%%%% SPACES %%%%%%%%%%%%%%%%%
\newcommand{\qandq}{ \quad \text{and} \quad }
\newcommand{\qqandqq}{ \qquad \text{and} \qquad }
\newcommand{\qifq}{ \quad \text{if} \quad }
\newcommand{\qqifqq}{ \qquad \text{if} \qquad }
\newcommand{\qwhereq}{ \quad \text{where} \quad }
\newcommand{\qqwhereqq}{ \qquad \text{where} \qquad }
\newcommand{\qwithq}{ \quad \text{with} \quad }
\newcommand{\qqwithqq}{ \qquad \text{with} \qquad }
\newcommand{\qforq}{ \quad \text{for} \quad }
\newcommand{\qqforqq}{ \qquad \text{for} \qquad }
\newcommand{\qqsinceqq}{ \qquad \text{since} \qquad }
\newcommand{\qsinceq}{ \quad \text{since} \quad }
\newcommand{\qarrq}{\quad\Longrightarrow\quad}
\newcommand{\qqarrqq}{\quad\Longrightarrow\quad}
\newcommand{\qiffq}{\quad\Longleftrightarrow\quad}
\newcommand{\qqiffqq}{\qquad\Longleftrightarrow\qquad}
\newcommand{\qsubjq}{ \quad \text{subject to} \quad }
\newcommand{\qqsubjqq}{ \qquad \text{subject to} \qquad }
\newcommand{\qobjq}[1]{\quad \text{#1} \quad}
\newcommand{\qqobjqq}[1]{\quad \text{#1} \quad}



% Restrictions
% Exemple: $ \rest{f}{ \Z } $

\newlength{\restsubwidth}
\newlength{\restsubheight}
\newlength{\restsubmoreheight}
\setlength{\restsubmoreheight}{4pt}
\newcommand{\rest}[2]{%
        \settowidth{\restsubwidth}{\ensuremath{#2}}
        \settoheight{\restsubheight}{\ensuremath{{}_{#2}}}
        \ensuremath{{#1\hskip 0.5pt}_{\vrule\kern2pt\parbox[b][%
        4pt][b]{\the\restsubwidth}{%
                        \ensuremath{{}_{#2}}}}}
        }

\addbibresource{biblio.bib}

\begin{document}

\section{Online Sinkhorn}

The Sinkhorn objective rewrites
\begin{equation}
    \max_{f, g \in \Cc(\Xx)} \dotp{f}{\alpha} + \dotp{g}{\beta}
     - \epsilon \dotp{\alpha \otimes \beta}{\exp(-\frac{f \oplus g - C}{\epsilon})
     }
\end{equation}
We perform the following change of variable $\mu = \alpha e^{f / \varepsilon}$,
$\nu = \beta e^{g / \epsilon}$, to obtain the equivalent problem, in $\Mm^+(\Xx)$
\begin{equation}
    \min_{\mu,\nu \in \Mm^+(\Xx)} \KL(\alpha|\mu) + \KL(\beta|\mu) + 
    \epsilon \dotp{\mu \otimes \nu}{\exp(-\frac{C}{\epsilon})} \triangleq f(\mu, \nu).
\end{equation}

The problem is not jointly convex, but convex in $\mu$ and $\nu$.
We may approach this problem from a game point of view of finding
 a local Nash equilibrium $(\mu^\star,\nu^\star)$ such that
\begin{align}\label{eq:Nash}
    \mu^\star &= \argmin_{\mu \in V(\mu^\star)} \Ff(\mu, \nu^\star) \\
    \nu^\star &= \argmin_{\nu \in V(\nu^\star)} \Ff(\mu^\star, \nu),
\end{align}
where $V$ are open sets. Such a formalism is useful as results on mirror descent
convergence in multi-agent setting exist for this problem. To solve
\eqref{eq:Nash}, we need to define distance generating functions to move back
and forth from $\mu$ and $\nu$ and their dual form. We define
\begin{align}
    \omega_\alpha(\mu) &\triangleq \KL(\alpha | \mu) \\
    \omega_\beta(\nu) &\triangleq \KL(\beta | \nu) \\
\end{align},
associated the the mirror maps
\begin{align}
    \nabla_\mu \omega_\alpha(\mu) &= 
    - \frac{\d\alpha}{\d\mu} \qquad \big( = - \exp(-f / \epsilon) \big), \\
    \nabla_\nu \omega_\beta(\nu) &= 
    - \frac{\d\beta}{\d\nu} \qquad \big( = - \exp(-g / \epsilon) \big),
\end{align}
with inverse
\begin{align}
    \nabla_\mu \omega^\star_\alpha(p) &= - \frac{\alpha}{p}, \\
    \nabla_\nu \omega^\star_\beta(q) &= - \frac{\beta}{q}.
\end{align}
\paragraph{Algorithm.} Let us consider the simple simultaneous mirror descent setting, where we build
the sequence of iterate $(\mu_t, \nu_t)_t$. It is easy to shows that if we start
from $\mu_0 \gg \alpha$ and $\nu_0 \gg \beta$, the iterates will remain
absolutely continuous with respect to $\alpha$ and $\beta$. We will therefore
write $\mu_t = \alpha e^{f_t / \varepsilon}$, $\nu_t = \beta e^{g_t /
\epsilon}$. The mirror descent iterations rewrite (for $\mu$)
\begin{equation}
    \mu_{t+1} = \frac{\alpha}{e^{-f_t / \epsilon} + \eta \nabla_\mu \Ff(\mu_t, \nu_t)},
\end{equation}
with $\nabla_\mu \Ff(\mu_t, \nu_t) = - \exp(-\frac{f_t}{\epsilon})+ \epsilon
\int_y \exp(\frac{g(y)-C(\cdot, y)}{\epsilon})\d\beta(y)$. We therefore have the
following update rules
\begin{align}
    \exp(-\frac{f_{t+1}}{\epsilon}) &=
     (1 - \eta) \exp(-\frac{f_{t}}{\epsilon}) 
     + \eta \EE_\beta[\epsilon \exp( \frac{g_t(y)-C(\cdot, y)}{\epsilon})], \\
     \exp(-\frac{g_{t+1}}{\epsilon}) &=
     (1 - \eta) \exp(-\frac{g_{t}}{\epsilon}) 
     + \eta \EE_\alpha[\epsilon \exp( \frac{f_t(x)-C(x, \cdot)}{\epsilon})].
\end{align}
Assuming we sample $\hat \beta_t = \sum_{i=1}^n b_{i,t} \delta_{y_{i,t}}$ and
$\hat \alpha_t = \sum_{i=1}^n a_{i,t} \delta_{x_{i,t}}$, we can approximate the
expectations above, and expect, with decreasing step-sizes to achieve
convergence.

Some variants (more likely to converge better) may be considered. The alternated variant writes
\begin{align}
    \exp(-\frac{f_{t+1}}{\epsilon}) &=
     (1 - \eta) \exp(-\frac{f_{t}}{\epsilon}) 
     + \eta \EE_\beta[\epsilon \exp( \frac{g_t(y)-C(\cdot, y)}{\epsilon})], \\
     \exp(-\frac{g_{t+1}}{\epsilon}) &=
     (1 - \eta) \exp(-\frac{g_{t}}{\epsilon}) 
     + \eta \EE_\alpha[\epsilon \exp( \frac{f_{t+1}(x)-C(x, \cdot)}{\epsilon})],
\end{align}
and the extrapolated version
\begin{align}
    \exp(-\frac{f_{t+1/2}}{\epsilon}) &=
     (1 - \eta) \exp(-\frac{f_{t}}{\epsilon}) 
     + \eta \EE_\beta[\epsilon \exp( \frac{g_t(y)-C(\cdot, y)}{\epsilon})], \\
     \exp(-\frac{g_{t+1/2}}{\epsilon}) &=
     (1 - \eta) \exp(-\frac{g_{t}}{\epsilon}) 
     + \eta \EE_\alpha[\epsilon \exp( \frac{f_t(x)-C(x, \cdot)}{\epsilon})], \\
     \exp(-\frac{f_{t+1}}{\epsilon}) &=
     \exp(-\frac{f_{t}}{\epsilon}) - \eta \exp(-\frac{g_{t + 1 / 2}}{\epsilon})
     + \eta \EE_\beta[\epsilon \exp( \frac{g_{t+1/2}(y)-C(\cdot, y)}{\epsilon})], \\
     \exp(-\frac{g_{t+1}}{\epsilon}) &=
     \exp(-\frac{g_{t}}{\epsilon}) - \eta \exp(-\frac{g_{t + 1 / 2}}{\epsilon})
     + \eta \EE_\alpha[\epsilon \exp( \frac{f_{t+1/2}(x)-C(x, \cdot)}{\epsilon})].
\end{align}
\paragraph{Computations.} In the simple simultaneous case, we can track $f_t$ in memory by the following representation
\begin{align}
    f_t(\cdot) &= - \epsilon \log \sum_{s=0}^t w_{t,s} \sum_{j=1}^n b_{s, j} 
    \exp\Big(g_{s-1}(y_{s,j}) - \frac{C(\cdot, y_{s, j})}{\epsilon}\Big) \\
    g_t(\cdot) &= - \epsilon \log \sum_{s=0}^t w_{t,s} \sum_{i=1}^n a_{s, i} 
    \exp\Big(f_{s-1}(x_{s, i}) - \frac{C(x_{s,i}, \cdot)}{\epsilon}\Big),
\end{align}
with $w_{t,s} = \eta (1-\eta)^{t-s}$ for $1 \leq s \leq t$, $w_{t,0} =
(1-\eta)^{t}$, and we set $g_{-1} = f_{-1} = 0$. The weights are a bit more
complex is $\eta$ depends on time.

The alternated version sets
\begin{equation}
    g_t(\cdot) = - \epsilon \log \sum_{s=0}^t w_{t,s} \sum_{i=1}^n a_{s, i} 
    \exp\Big(f_{s}(x_{s, i}) - \frac{C(x_{s,i}, \cdot)}{\epsilon}\Big),
\end{equation}

Setting $q_{t,s,i} = w_{t,s} b_{s, j} \exp(g_{s-1}(y_{s,
j}))$ and $p_{t,s,i} = w_{t,s} b_{s, j} \exp(f_{s-1}(x_{s, j})/ \epsilon)$, we can derive
simple update rules for $p$ and $q$:
\begin{equation}
    p_{t,t, i} = \eta b_{t,j} \exp(f_{t-1}(x_{t, j}),
    \qquad \foralls s < t,\quad p_{t,s, i} = (1- \eta) p_{t-1,s, i}
\end{equation}


\section{Analysis}

Bregman divergence associated to $\phi$ (désolé j'ai changé de notation).
\begin{equation}
    d_{\phi}(f_2|f_1) = \dotp{\alpha}{\exp(\frac{f_2-f_1}{\varepsilon})
     - 1 - \frac{f_2-f_1}{\varepsilon}}
      \geq \dotp{\alpha}{\big(\frac{f_2-f_1}{2 \varepsilon}\big)^2}
\end{equation}

For convergence of MD on $\min f(x)$ with mirror map $\phi$, we need to show, according to Gabriel, Jalal and Kelvin
\begin{equation}
    \mu d_\phi(x_2|x_1 )\leq d_f(x_2|x_1) \leq L d_\phi(x_2|x_1).
\end{equation}
Can we use that here ? Beware that we are in an alternated setting


\subsection{Sketch of proof}

See proof of Th2 in Ya Ping's paper.

\subsubsection{General proof}

By using the dual iteration and the three point property (normally holds by def of $D_{\alpha}$ and $D_{\beta}$):
\begin{align*}
\ps{\mu_t-\mu}{-\nabla_{\mu} \Ff(\mu_t,\nu_t)}&=\frac{1}{\eta}\ps{\mu_t-\mu}{\nabla_{\mu}w_{\alpha}(\mu_{t+1})-\nabla_{\mu}w_{\alpha}(\mu_t)}\\
&=\frac{1}{\eta}[D_{w_{\alpha}}(\mu,\mu_{t})-D_{w_{\alpha}}(\mu,\mu_{t+1})+D_{w_{\alpha}}(\mu_t,\mu_{t+1})]
\end{align*}
Suppose we can show (TO DO): 
\begin{align}\label{eq:bound_gradient}
D_{w_{\alpha}}(\mu_t,\mu_{t+1})\le \eta^2 M^2
\end{align}
Then we have:
\begin{align*}
\frac{1}{T}\sum_{t=1}^T \ps{\mu_t-\mu}{-\nabla_{\mu} \Ff(\mu_t,\nu_t)}&= \sum_{t=1}^T \frac{1}{\eta}[D_{w_{\alpha}}(\mu,\mu_{t})-D_{w_{\alpha}}(\mu,\mu_{t+1})+D_{w_{\alpha}}(\mu_t,\mu_{t+1})]\\
\le \frac{D_{w_{\alpha}}(\mu,\mu_1)}{\eta}+\eta^2 M^2 T
\end{align*}
Similarly:
\begin{align*}
\frac{1}{T}\sum_{t=1}^T \ps{\nu_t-\nu}{-\nabla_{\nu} \Ff(\mu_t,\nu_t)}&
\le \frac{D_{w_{\beta}}(\mu,\mu_1)}{\eta}+\eta^2 M^2 T
\end{align*}
Summing up the two previous equations and replacing $(\mu,\nu)$ by $(\mu*,\nu*)$, we get:
\begin{equation}
\frac{1}{T}\sum_{t=1}^T \ps{\mu_t-\mu*}{-\nabla_{\mu} \Ff(\mu_t,\nu_t)} + \ps{\nu_t-\nu*}{-\nabla_{\nu} \Ff(\mu_t,\nu_t)}\le \frac{D_0}{\eta}+2\eta^2 M^2 T
\end{equation}
where $D_0=D_{w_{\alpha}}(\mu^*,\mu_1)+D_{w_{\beta}}(\nu^*,\nu_1)$.\\

\noindent Then, by optimality of $\mu^*$ and convexity of $\Ff$:
\begin{multline*}
\Ff(\mu_t,\nu_t)-\Ff(\mu^*,\nu^*)\le \Ff(\mu_t,\nu_t)-\Ff(\mu^*,\nu_t)
\le \ps{\mu^*-\mu_t}{\nabla_{\mu}\Ff(\mu_t,\nu_t)}= \ps{\mu_t-\mu^*}{-\nabla_{\mu}\Ff(\mu_t,\nu_t)}
\end{multline*}
Hence:
\begin{equation}
\frac{1}{T}\sum_{t=1}^T  
\Ff(\mu_t,\nu_t)-\Ff(\mu^*,\nu^*) \le \frac{D_0}{\eta}+2\eta^2 M^2 T
\end{equation}

\subsubsection{Tricky part}
Now let's try to prove Equation \ref{eq:bound_gradient}. 
What we need is 
\begin{itemize}
	\item $D_{w_{\alpha}}(\mu,\nu)=D_{w_{\alpha}^*}(\nabla_{\mu}w_{\alpha}(\mu), \nabla_{\mu}w_{\alpha}(\nu))$ (eq A.13 in Ya Ping's paper)
	\item  \textit{relative smoothness of $w_{\alpha}^*$ wrt $\|.\|_{\infty}$} (eq A.11 and A.12 in Ya Ping's paper)
\end{itemize}
 If it's true:
\begin{align*}
D_{w_{\alpha}}(\mu_t,\mu_{t+1}) = D_{w_{\alpha}^*}(\nabla_{\mu}w_{\alpha}(\mu_t), \nabla_{\mu}w_{\alpha}(\mu_{t+1}))\le \| \nabla_{\mu}w_{\alpha}(\mu)- \nabla_{\mu}w_{\alpha}(\nu)\|^2_{\infty}\\
=\| \exp(-\frac{f_{t+1}}{\epsilon})- \exp(-\frac{f_t}{\epsilon})\|^2_{\infty} = \eta^2 \| \nabla \Ff_{\mu}(\mu_t,\nu_t)\|^2_{\infty}
\end{align*}
and 
\begin{align*}
\|\nabla \Ff_{\mu}(\mu_t,\nu_t)\|^2_{\infty}\le ?
\end{align*}

WIP
\begin{equation}
    D_{w_{\alpha}}(\mu_t,\mu_{t+1}) \geq
    \Vert \log \frac{\mathrm d \mu_{t+1}}{\mathrm d \mu_{t}} \Vert_\alpha^2
\end{equation}

We can show
\begin{align}
    D_{\omega_\alpha}(\mu_t | \mu_{t+1}) = \eta^2
    \dotp{\alpha}{1 - \exp(\frac{f_t - \hat f_{t+1}}{\epsilon})},
    = \eta^2 (1 - \dotp{\alpha}{\nabla_\mu \Ff(\mu_t, \nu_t)})
    \\ \hat f_{t+1}(\cdot) 
    = - \varepsilon \log \int_y \exp(\frac{g_t(y) - C(\cdot, y)}{\varepsilon}) \mathrm d \beta(y).
\end{align}
Avec Sinkhorn sans bruit $f_t - \hat f_{t+1}$ va rester tranquille.

\section{Proof of convergence}

We want to solve
\begin{equation}
    \min_{\mu \in \Mm^+(\Xx), \nu \in \Mm^+(\Xx)} 
    F(\mu, \nu) \triangleq \KL(\alpha | \mu) + \KL(\beta | \nu) + \dotp{\mu \otimes \nu}{\exp(-C)}
\end{equation}

Let's write $x = (\mu, \nu)$  and $F(\mu, \nu) = F(x)$ the objective. We define the iterates $x_{t} = (\mu_t, \nu_t)$, $x_{t+1/2} = (\mu_{t+1}, \nu_t)$, 
$x_{t} = (\mu_{t+1}, \nu_{t+1})$. First note that we have
\begin{equation}
    D_{F(\mu, \cdot)} = D_{\omega_\alpha(\cdot)}\qquad D_{F(\cdot, \nu)} = D_{w_\beta(\cdot)},
\end{equation}
so that at every iteration, we perform a mirror step with a function that is both 1-relatively smooth and 1-relatively strongly convex.

Let $\nu$ be fixed, and let us define $F_\nu(\cdot) = F(\cdot, \nu)$. From the
smoothness of $F_\nu(\cdot)$ and from its convexity we have, for all 
$\mu_x, \mu_y, \mu_z \gg \alpha$,
\begin{align}
    F(\mu_x, \nu) &\leq F(\mu_y, \nu)
     + \dotp{\nabla_\mu F(\mu_y, \nu)}{\mu_x- \mu_y} + L D_{\omega_\alpha}(\mu_x, \mu_y), \\
    F(\mu_y, \nu) &\leq F(\mu_z, \nu) + \dotp{\nabla_\mu F(\mu_y, \nu)}{\mu_y - \mu_z}
\end{align}
Combining both, we obtain
\begin{align}
    F(\mu_x, \nu) &\leq F(\mu_z, \nu) + 
    \dotp{\nabla_\mu F(\mu_y, \nu)}{\mu_x - \mu_z} + L D_{\omega_\alpha}(\mu_x, \mu_y) \\
    \dotp{\nabla F(\mu_y, \nu)}{\mu_x - \mu_z} 
    &\geq F(\mu_x, \nu) - F(\mu_z, \nu) - L D_{\omega_\alpha}(\mu_x, \mu_y).
\end{align}
We now use the three point propery:
\begin{equation}
    D_{\omega_\alpha}(\mu_z, \mu_y) - D_{\omega_\alpha}(\mu_z, \mu_x) 
    - D_{\omega_\alpha}(\mu_x, \mu_y) 
    = \dotp{\nabla \omega_\alpha(\mu_x) - \nabla \omega_\alpha(\mu_y)}{\mu_z - \mu_x},
\end{equation}
replacing $\mu_y = \mu_{k}, \mu_x = \mu_{k+1}$, to obtain
\begin{align}
    D_{\omega_\alpha}(\mu_z, \mu_k) - D_{\omega_\alpha}(\mu_z, \mu_{k+1}) 
    - D_{\omega_\alpha}(\mu_{k+1}, \mu_k) &=
     \eta_k \dotp{\nabla_\mu F(\mu_k, \nu)}{\mu_{k+1} - \mu_z} \\
    &\geq \eta_k (F(\mu_{k+1}, \nu) - F(\mu_z, \nu_k)))
     - L \eta_k D_{\omega_\alpha}(\mu_{k+1},\mu_k).
\end{align}
Hence, mimicking the derivation for $\nu$, using the gradient update step
\begin{align}
    \eta_k (F(\mu_{k+1},\nu_{k+1}) - F(\mu_{k+1}, \nu_z)) &\leq 
    D_{\omega_\beta}(\nu_z, \nu_k) - D_{\omega_\beta}(\nu_z, \nu_{k+1}) 
    - (1 - \eta_k L) D_{\omega_\beta}(\nu_{k+1}, \nu_k). \\
    \eta_k (F(\mu_{k+1},\nu_k) - F(\mu_z, \nu_k)) &\leq 
    D_{\omega_\alpha}(\mu_z, \mu_k) - D_{\omega_\alpha}(\mu_z, \mu_{k+1}) 
    - (1 - \eta_k L) D_{\omega_\alpha}(\mu_{k+1}, \mu_k)
\end{align}
Setting $\mu_z = \mu_{k}$, $\nu_z = \nu_k$, we obtain a descent lemma.
\begin{equation}
    F(\mu_{k+1}, \nu_{k+1}) \leq F(\mu_{k+1}, \nu_{k}) \leq F(\mu_{k}, \nu_{k}).
\end{equation}
Therefore $(F())
% Choosing a stationary point $(\mu^\star, \nu^\star)$, setting $\nu_z = \nu^\star$,
% $\mu_z = \mu^\star$, and summing both equations over $k$, we obtain, using the
% fact that decreasing
% \begin{equation}
%     F(\mu_{K+1}, \nu_{K+1}) + 
%     F(\mu^\star, \nu_{K+1}) - 2 F^\star \leq\frac{1}{\sum_{k=1}^K \eta_k} \sum_{k=1}^K \eta_k 
%     \big( F(\mu_{k+1}, \nu^\star)
%      + F(\mu^\star, \nu_{k+1})  - 2 F^\star \big)
%      \leq \frac{D_{\omega_\alpha}(\mu^\star, \mu_1)
%      + D_{\omega_\beta}(\nu^\star, \nu_1)}{\sum_{k=1}^K \eta_k}
% \end{equation}
% Hence
% \begin{equation}
%     F(\mu_{K+1}, \nu^\star) + F(\mu^\star, \nu_{K+1}) \to 2 F^\star.
% \end{equation}

\printbibliography

\end{document}
