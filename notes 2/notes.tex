\documentclass[a4paper, 10pt]{article}
%!TEX root = article.tex
%%%%%%%%%%%%%%%%%%%%%% INCLUDES %%%%%%%%%%%%%%%%
% Copyright (c) 2007 Gabriel Peyre
% Modfied 2018 Arthur Mensch
\usepackage[utf8]{inputenc}
\usepackage[T1]{fontenc}

 

% Bold math
\usepackage{bm}
\usepackage{dsfont}
\usepackage{csquotes}

% Micro-typing
\usepackage{xspace}
\usepackage{nicefrac}
\usepackage[super]{nth}
\usepackage{microtype}

\usepackage[english]{babel}

% Compact itemize
\usepackage{enumitem}

% Tables
\usepackage{booktabs}
\usepackage{makecell}


\usepackage{graphicx}
\usepackage{subcaption}

% For colors
\usepackage{xcolor}

% Algorithm
\usepackage{algorithm}
\usepackage[noend]{algpseudocode}
\algrenewcommand{\algorithmiccomment}[1]{$\vartriangleright$ #1}
\algrenewcommand{\algorithmicreturn}{\textbf{Return: }}
\algnewcommand\algorithmicinput{\textbf{Input: }}
\algnewcommand\Input{\State \algorithmicinput}

% Hyper reference
\usepackage[colorlinks=true,bookmarks=true,linkcolor=blue,
urlcolor=blue,citecolor=blue,breaklinks=true]{hyperref}

\usepackage{url}

\makeatletter
\addto\extrasenglish{%
\renewcommand{\appendixautorefname}{App.}%
\renewcommand{\figureautorefname}{Fig.}%
\renewcommand{\tableautorefname}{Table}%
\renewcommand{\sectionautorefname}{\S\@gobble}%
\renewcommand{\subsectionautorefname}{\S\@gobble}%
\renewcommand{\subsubsectionautorefname}{\S\@gobble}%
}
\makeatother

% \makeatletter
% \renewbibmacro*{textcite}{%
%   \iffieldequals{namehash}{\cbx@lasthash}
%     {\usebibmacro{cite:comp}}
%     {\usebibmacro{cite:dump}%
%      \iffirstcitekey
%        {}
%        {\textcitedelim}%
%      \usebibmacro{cite:init}%
%      \ifnameundef{labelname}
%        {\printfield[citetitle]{labeltitle}}
%        {\printnames{labelname}}%
%      \mkbibsuperscript{%
%      \usebibmacro{cite:comp}}%
%      \stepcounter{textcitecount}%
%      \savefield{namehash}{\cbx@lasthash}}}

% \DeclareCiteCommand{\textcite}
%   {\let\cbx@tempa=\empty
%    \undef\cbx@lasthash
%    \iffieldundef{prenote}
%      {}
%      {\BibliographyWarning{Ignoring prenote argument}}%
%    \iffieldundef{postnote}
%      {}
%      {\BibliographyWarning{Ignoring postnote argument}}}
%   {\usebibmacro{citeindex}%
%    \usebibmacro{textcite}}
%   {}
%   {}
% \makeatother

% Float parameters, for more full pages.
% From Gaël Varoquaux
\renewcommand{\topfraction}{0.9}	% max fraction of floats at top
\renewcommand{\bottomfraction}{0.8}	% max fraction of floats at bottom
\renewcommand{\textfraction}{0.07}	% allow minimal text w. figs
%   Parameters for FLOAT pages (not text pages):
\renewcommand{\floatpagefraction}{0.6}	% require fuller float pages
%    % N.B.: floatpagefraction MUST be less than topfraction !!
\renewcommand{\dbltopfraction}{.95}  % double page floats at the top
\renewcommand{\dblfloatpagefraction}{.6}
\setcounter{totalnumber}{4}
\setcounter{topnumber}{3}
\setcounter{dbltopnumber}{3}
\setcounter{bottomnumber}{2}

% Wide pages for supplementary figures
\usepackage{changepage}                 % adjust margins for selected portions
% wide page for side by side figures, tables, etc
\newlength{\offsetpage}
\setlength{\offsetpage}{3.0cm}
\newenvironment{widepage}{\begin{adjustwidth}{-\offsetpage}{-\offsetpage}%
    \addtolength{\textwidth}{2\offsetpage}}%
{\end{adjustwidth}}


% MATHS (AMS)
\usepackage{amsmath}
\usepackage{amsfonts} 
\usepackage{amssymb}
\usepackage{mathrsfs}
\usepackage{mathtools} % for psmallmatrix
\mathtoolsset{showonlyrefs}

% Algorithm
%\usepackage{algorithm}
\usepackage[noend]{algpseudocode}
%\algrenewcommand{\algorithmiccomment}[1]{$\vartriangleright$ #1}
%\algrenewcommand{\algorithmicreturn}{\textbf{Return: }}
%\algnewcommand\algorithmicinput{\textbf{Input: }}
%\algnewcommand\Input{\State \algorithmicinput}

\usepackage{footnote} % for footnotes
\makesavenoteenv{algorithm}

\usepackage{amsthm}

\newtheorem{theorem}{Theorem}
\newtheorem{proposition}{Proposition}
\newtheorem{definition}{Definition}
\newtheorem{corollary}{Corollary}
\newtheorem{lemma}{Lemma}
\newtheorem{remark}{Remark}
\newtheorem{example}{Example}
\newtheorem{assumption}{Assumption}

\newtheorem{subassumptioninner}{Assumption}
\newenvironment{subassumption}[1]{%
  \renewcommand\thesubassumptioninner{#1}%
  \subassumptioninner
}{\endsubassumptioninner}

\newcommand{\definitionautorefname}{Def.}%
\newcommand{\propositionautorefname}{Prop.}%
\newcommand{\corollaryautorefname}{Corollary}%
\newcommand{\assumptionautorefname}{Ass.}%
\newcommand{\subassumptioninnerautorefname}{Ass.}%
\newcommand{\algorithmautorefname}{Alg.}%
\newcommand{\lemmaautorefname}{Lemma}
\newcommand{\remarkautorefname}{Rmk.}

\newcommand{\todo}[1]{{\color{red} TODO: #1}}
\newcommand{\enscond}[2]{ \left\{ #1 \;:\; #2 \right\} }

\newcommand{\Ctrans}[2]{T_{#2}(#1)}


%!TEX root = article.tex
%% Symboles avec double lignes
\newcommand{\NN}{\mathbb{N}}
\newcommand{\CC}{\mathbb{C}}
\newcommand{\GG}{\mathbb{G}}
\newcommand{\LL}{\mathbb{L}}
\newcommand{\PP}{\mathbb{P}}
\renewcommand{\SS}{\mathbb{S}}
\newcommand{\QQ}{\mathbb{Q}}
\newcommand{\RR}{\mathbb{R}}
\newcommand{\VV}{\mathbb{V}}
\newcommand{\ZZ}{\mathbb{Z}}
\newcommand{\FF}{\mathbb{F}}
\newcommand{\KK}{\mathbb{K}}
\newcommand{\TT}{\mathbb{T}}
\newcommand{\UU}{\mathbb{U}}
\newcommand{\EE}{\mathbb{E}}
\newcommand{\XX}{\mathbb{X}}
\newcommand{\YY}{\mathbb{Y}}
%% Symboles arrondis
\newcommand{\Aa}{\mathcal{A}}
\newcommand{\Bb}{\mathcal{B}}
\newcommand{\Cc}{\mathcal{C}}
\newcommand{\Dd}{\mathcal{D}}
\newcommand{\Ee}{\mathcal{E}}
\newcommand{\Ff}{\mathcal{F}}
\newcommand{\Gg}{\mathcal{G}}
\newcommand{\Hh}{\mathcal{H}}
\newcommand{\Ii}{\mathcal{I}}
\newcommand{\Jj}{\mathcal{J}}
\newcommand{\Kk}{\mathcal{K}}
\newcommand{\Ll}{\mathcal{L}}
\newcommand{\Mm}{\mathcal{M}}
\newcommand{\Nn}{\mathcal{N}}
\newcommand{\Oo}{\mathcal{O}}
\newcommand{\Pp}{\mathcal{P}}
\newcommand{\Qq}{\mathcal{Q}}
\newcommand{\Rr}{\mathcal{R}}
\newcommand{\Ss}{\mathcal{S}}
\newcommand{\Tt}{\mathcal{T}}
\newcommand{\Uu}{\mathcal{U}}
\newcommand{\Vv}{\mathcal{V}}
\newcommand{\Ww}{\mathcal{W}}
\newcommand{\Xx}{\mathcal{X}}
\newcommand{\Yy}{\mathcal{Y}}
\newcommand{\Zz}{\mathcal{Z}}
\newcommand{\KL}{\textrm{KL}}

% matrices
\newcommand{\A}{\bm{A}}
\newcommand{\B}{\bm{B}}
% \newcommand{\C}{\bm{C}}
\newcommand{\D}{\bm{D}}
%\newcommand{\E}{\bm{E}}
\newcommand{\F}{\bm{F}}
% \newcommand{\G}{\bm{G}}
\newcommand{\Hz}{\bm{H}}
\newcommand{\I}{\bm{I}}
\newcommand{\J}{\bm{J}}
\newcommand{\K}{\bm{K}}
\newcommand{\Lz}{\bm{L}}
\newcommand{\M}{\bm{M}}
\newcommand{\N}{\bm{N}}
\newcommand{\Oz}{\bm{O}}
\newcommand{\Pz}{\bm{P}}
\newcommand{\Q}{\bm{Q}}
\newcommand{\R}{\bm{R}}
\newcommand{\Sz}{\bm{S}}
\newcommand{\T}{\bm{T}}
% \newcommand{\U}{\bm{U}}
\newcommand{\V}{\bm{V}}
\newcommand{\W}{\bm{W}}
\newcommand{\X}{\bm{X}}
\newcommand{\Y}{\bm{Y}}
\newcommand{\Z}{\bm{Z}}

% Vectors
\renewcommand{\a}{\bm{a}}
\renewcommand{\b}{\bm{b}}
\renewcommand{\c}{\bm{c}}
\renewcommand{\d}{\textrm{d}}
\newcommand{\dz}{\bm{d}}
\newcommand{\e}{\bm{e}}
% \newcommand{\f}{\bm{f}}
\newcommand{\g}{\bm{g}}
\newcommand{\h}{\bm{h}}
\renewcommand{\i}{\bm{i}}
\renewcommand{\j}{\bm{j}}
\renewcommand{\k}{\bm{k}}
\renewcommand{\l}{\bm{l}}
\newcommand{\m}{\bm{m}}
\newcommand{\n}{\bm{n}}
\renewcommand{\o}{\bm{o}}
\newcommand{\p}{\bm{p}}
\newcommand{\q}{\bm{q}}
\renewcommand{\r}{\bm{r}}
\newcommand{\s}{\bm{s}}
\renewcommand{\t}{\bm{t}}
\renewcommand{\u}{\bm{u}}
\renewcommand{\v}{\bm{v}}
\newcommand{\w}{\bm{w}}
\newcommand{\x}{\bm{x}}
\newcommand{\y}{\bm{y}}
\newcommand{\z}{\bm{z}}

%commands added by Carles
\newcommand{\xx}{\mathbf{x}}
\newcommand{\yy}{\mathbf{y}}
\newcommand{\zz}{\mathbf{z}}
\newcommand{\hmu}{\hat{\mu}}

% greek
\newcommand{\al}{\alpha}
\newcommand{\la}{\lambda}
\newcommand{\ga}{\gamma}
\newcommand{\Ga}{\Gamma}
\newcommand{\La}{\Lambda}
\newcommand{\si}{\sigma}
%\newcommand{\Si}{\Sigma}
\newcommand{\be}{\beta}
\newcommand{\de}{\delta}
\newcommand{\De}{\Delta}
\renewcommand{\phi}{\varphi}
\renewcommand{\th}{\theta}
\newcommand{\om}{\omega}
\newcommand{\Om}{\Omega}

% hat, tilde
\newcommand{\hf}{\hat f}
\newcommand{\wtf}{\tilde f}
\newcommand{\tx}{\tilde x}
\newcommand{\ta}{\tilde a}
\newcommand{\tb}{\tilde b}
\newcommand{\ty}{\tilde y}
\newcommand{\tu}{\tilde u}
\newcommand{\tv}{\tilde v}
\newcommand{\tga}{\tilde \ga}
\newcommand{\tf}{\wt{f}}

\newcommand{\Simplex}{\triangle}
\newcommand{\errnash}{\textrm{Err}_N}

% full, alternate, random
\newcommand{\full}{\textnormal{full}}
\newcommand{\cyc}{\textnormal{cyc}}
\newcommand{\rdm}{\textnormal{rand}}

%%%%%%%%%%%%%%% MATHS OPERATORS %%%%%%%%%%%%%%%%%
\newcommand{\ins}[1]{\mathrm{#1}}

\DeclareMathOperator{\realp}{\Rr e}
\DeclareMathOperator{\imagp}{\Ii m}
\DeclareMathOperator{\Ker}{Ker}
\DeclareMathOperator{\Hom}{Hom}
\DeclareMathOperator{\End}{End}
\DeclareMathOperator{\tr}{tr}
\DeclareMathOperator{\Tr}{Tr}
\DeclareMathOperator{\Supp}{Supp}
\DeclareMathOperator{\Sign}{Sign}
\let\Im\relax
\DeclareMathOperator{\Im}{Im}
% \let\span\relax
%\DeclareMathOperator{\span}{span}
\DeclareMathOperator{\Corr}{Corr}
\DeclareMathOperator{\sign}{sign}
\DeclareMathOperator{\supp}{supp}

\DeclareMathOperator{\cas}{cas}
\DeclareMathOperator{\sinc}{sinc}
\DeclareMathOperator{\cotan}{cotan}
\DeclareMathOperator{\Card}{Card}
\DeclareMathOperator{\GCD}{GCD}
\DeclareMathOperator{\grad}{grad}
\DeclareMathOperator{\Diag}{Diag}
\DeclareMathOperator{\rank}{rank}
\DeclareMathOperator{\conv}{conv}
\DeclareMathOperator{\interop}{int}

\newcommand{\ps}[2]{\langle #1,#2\rangle}
\newcommand{\E}[1]{\mathbb{E}\left[#1\right] }

% regularity
\newcommand{\Calpha}{\mathrm{C}^\al}
\newcommand{\Cbeta}{\mathrm{C}^\be}
\newcommand{\Cal}{\text{C}^\al}
\newcommand{\Ctwo}{\text{C}^{2}}
\newcommand{\Calt}[1]{\text{C}^{#1}}
\newcommand{\Cder}[1]{\mathscr{C}^{#1}}
% Lp spaces
\newcommand{\lone}{\ell^1}
\newcommand{\ltwo}{\ell^2}
\newcommand{\linf}{\ell^\infty}
\newcommand{\Lone}{\text{\upshape L}^1}
\newcommand{\Ltwo}{\text{\upshape L}^2}
\newcommand{\Linf}{\text{\upshape L}^\infty}
\newcommand{\lzero}{\ell^0}
\newcommand{\lp}{\ell^p}
% circle
\newcommand{\Sun}{\text{S}^1}
% little space after forall
\newcommand{\foralls}{\forall \,}
%% for derivatives
\newcommand{\dd}{\ins{d}}
%% Use french comparaison operator
\renewcommand{\leq}{\leqslant}
\renewcommand{\geq}{\geqslant}


%%%%%%%%%%%%%%% MATHS CONSTRUCTS %%%%%%%%%%%%%%%

%% over-symbols
\newcommand{\ol}[1]{\overline{#1}}
\newcommand{\wh}[1]{\widehat{#1}}
\newcommand{\whwh}[1]{\hat{\hat{#1}}}
\newcommand{\wt}[1]{\widetilde{#1}}
%% partial derivatives
\newcommand{\pd}[2]{ \frac{ \partial #1}{\partial #2} }
\newcommand{\pdd}[2]{ \frac{ \partial^2 #1}{\partial #2^2} }
%% nice epsilon
\renewcommand{\epsilon}{\varepsilon}
%% Pour avoir un joli i pour les complexes
\renewcommand{\imath}{\mathrm{i}}
\newcommand{\interior}[1]{\ensuremath{\overset{\circ}{#1}}}

%% Legendre symbol
\newcommand{\legsymb}[2]{ \genfrac{(}{)}{}{}{#1}{#2} }
%% exposant pour les ordinaux
\newcommand{\ordin}[2]{ ${#1}^{ \text{#2} }$ }
%% Dot product and cross product
% \newcommand{\dotp}[2]{ \left\langle #1,\,#2 \right\rangle }
\newcommand{\dotp}[2]{\langle #1,\,#2\rangle}
\newcommand{\dotps}[2]{\langle #1,\,#2\rangle}
\newcommand{\crossp}[2]{ #1 \hat #2 }
\newcommand{\seg}[2]{\llbracket #1,\,#2 \rrbracket}
\newcommand{\brac}[1]{\left[#1\right]}
%\newcommand{\norm}[1]{|\!| #1 |\!|}
\newcommand{\norm}[1]{\left\| #1 \right\|}
\newcommand{\snorm}[1]{\| #1 \|}
\newcommand{\normb}[1]{\Big|\!\Big| #1 \Big |\!\Big|}
\newcommand{\normB}[1]{\left|\!\left| #1 \right|\!\right|}
\DeclareMathOperator{\diverg}{div}
\DeclareMathOperator{\Prox}{Prox}
\newcommand{\normT}[1]{\norm{#1}_{\text{T}}}
\newcommand{\normu}[1]{\norm{#1}_{1}}
\newcommand{\normi}[1]{\norm{#1}_{\infty}}
\newcommand{\normd}[1]{\norm{#1}_{2}}
\newcommand{\normz}[1]{\norm{#1}_{0}}
\newcommand{\abs}[1]{\left\lvert#1\right\rvert} % modified by Vincent
\newcommand{\absb}[1]{\Big| #1 \Big|}

%% Function definition
\newcommand{\func}[4]{ {\left\{  \begin{array}{ccc} #1 & \longrightarrow & #2 \\ #3 & \longmapsto & #4 \end{array}  \right.} }
% transpose
\newcommand{\transp}[1]{ {#1}^{\ins{T}} }
% l'identit�
\newcommand{\Id}{\ins{Id}}
% egal par d�finition
\newcommand{\eqdef}{\triangleq}

\DeclareMathOperator*{\argmin}{argmin}
\DeclareMathOperator*{\argmax}{argmax}
\DeclareMathOperator*{\interi}{int}
% \DeclareMathOperator*{\sup}{sup}

\DeclarePairedDelimiter{\ceil}{\lceil}{\rceil}
\DeclarePairedDelimiter{\floor}{\lfloor}{\rfloor}
\newcommand*\red{\color{red}}
\newcommand{\vardbtilde}[1]{\tilde{\raisebox{0pt}[0.85\height]{$\tilde{#1}$}}}
\usepackage{accents}
\newcommand{\dbtilde}[1]{\accentset{\approx}{#1}}

%% parenthesis
\newcommand{\pa}[1]{\left( #1 \right)}
\newcommand{\bpa}[1]{\big( #1 \big)}
\newcommand{\choice}[1]{ %
	\left\{ %
		\begin{array}{l} #1 \end{array} %
	\right. }
% ensembles
\newcommand{\set}[1]{ \{ #1 \} }
\newcommand{\condset}[2]{ \left\{ #1 \;;\; #2 \right\} }


%%%%%%%%%%%%%%% SPACES %%%%%%%%%%%%%%%%%
\newcommand{\qandq}{ \quad \text{and} \quad }
\newcommand{\qqandqq}{ \qquad \text{and} \qquad }
\newcommand{\qifq}{ \quad \text{if} \quad }
\newcommand{\qqifqq}{ \qquad \text{if} \qquad }
\newcommand{\qwhereq}{ \quad \text{where} \quad }
\newcommand{\qqwhereqq}{ \qquad \text{where} \qquad }
\newcommand{\qwithq}{ \quad \text{with} \quad }
\newcommand{\qqwithqq}{ \qquad \text{with} \qquad }
\newcommand{\qforq}{ \quad \text{for} \quad }
\newcommand{\qqforqq}{ \qquad \text{for} \qquad }
\newcommand{\qqsinceqq}{ \qquad \text{since} \qquad }
\newcommand{\qsinceq}{ \quad \text{since} \quad }
\newcommand{\qarrq}{\quad\Longrightarrow\quad}
\newcommand{\qqarrqq}{\quad\Longrightarrow\quad}
\newcommand{\qiffq}{\quad\Longleftrightarrow\quad}
\newcommand{\qqiffqq}{\qquad\Longleftrightarrow\qquad}
\newcommand{\qsubjq}{ \quad \text{subject to} \quad }
\newcommand{\qqsubjqq}{ \qquad \text{subject to} \qquad }
\newcommand{\qobjq}[1]{\quad \text{#1} \quad}
\newcommand{\qqobjqq}[1]{\quad \text{#1} \quad}

\newcommand{\carles}[1]{{\color{orange}[CD: #1]}}
\newcommand{\arthur}[1]{{\color{blue}[AM: #1]}}
\newcommand{\joan}[1]{{\color{green}[JB: #1]}}
\newcommand{\samy}[1]{{\color{red}[SJ: #1]}}


% Restrictions
% Exemple: $ \rest{f}{ \Z } $

\newlength{\restsubwidth}
\newlength{\restsubheight}
\newlength{\restsubmoreheight}
\setlength{\restsubmoreheight}{4pt}
\newcommand{\rest}[2]{%
        \settowidth{\restsubwidth}{\ensuremath{#2}}
        \settoheight{\restsubheight}{\ensuremath{{}_{#2}}}
        \ensuremath{{#1\hskip 0.5pt}_{\vrule\kern2pt\parbox[b][%
        4pt][b]{\the\restsubwidth}{%
                        \ensuremath{{}_{#2}}}}}
        }

% Document specific macros
\newcommand{\enscond}[2]{ \left\{ #1 \;:\; #2 \right\} }
\newcommand{\Ctrans}[2]{T_{#2}(#1)}
\newcommand{\var}{\text{var}}

\addbibresource{biblio.bib}
\addbibresource{SA.bib}

\begin{document}

\section{An online expectation minimization algorithm}

Define $\mu = \alpha \exp(f)$, $\nu = \beta \exp(g)$, $x = (\mu, \nu)$. We will change variables without warning in the following.
Define the Bregman divergence
\begin{align}
    D_{\alpha}(\mu, \mu_0) &= \dotp{\alpha}{\exp(f_0 - f) - 1 - (f_0 - f)} \\
    D_{\beta}(\nu, \nu_0) &= \dotp{\beta}{\exp(g_0 - g) - 1 - (g_0 - g)} \\
    D_{\alpha, \beta}(x, x_0) &= D_{\alpha}(\mu, \mu_0) + D_{\beta}(\nu, \nu_0)
\end{align}
We want to solve the objective
\begin{equation}
    \min_x \Ff(x) \triangleq \KL(\alpha, \mu) + \KL(\beta, \nu) + \dotp{\mu \otimes \nu}{\exp(-C)} - 1
\end{equation}
Define the prox objective
\begin{align}
    \Ll(x, x_t) &= 2 \Ff(x_t) + \dotp{\nabla \Ff(x_t)}{x - x_t} + D_{\alpha,\beta}(x, x_t) \\
    &= \EE_{\hat \alpha \sim \alpha, \hat \beta \sim \alpha}
    \Big[2 F(x_t) + \dotp{\nabla \Ff(x_t)}{x - x_t} + D_{\hat \alpha,\hat \beta}(x, x_t) \Big]
\end{align}
The Sinkhorn iterations then rewrites as
\begin{equation}
    x_{t+1} = \argmin_x \EE_{\hat \alpha, \hat \beta} \Ll_{\hat \alpha, \hat \beta}(x, x_t)
\end{equation}
and online Sinkhorn

\begin{equation}
    x_{t+1} = (1 - \eta_t) x_t + \eta_t \argmin_x \Ll_{\hat \alpha_t, \hat \beta_t}(x, x_t)
\end{equation}
Probably useless ?

\section{Variable mirror descent point of view}

Consider the objective
\begin{equation}
    \max_{f, g} \Ff(f, g) = \dotp{\alpha}{f} + \dotp{g, \beta} -
     \dotp{\alpha \otimes \beta}{\exp(f \oplus g - C)} + 1
\end{equation}
The gradient reads
\begin{equation}
    \nabla \Ff(f, g) = \Big(\alpha\big(1 - \exp(f - T_\beta(g))\big), \beta\big(1 - \exp(g - T_\alpha(f))\big)\Big)
     \in \Mm^+(\Xx^2)
\end{equation}
Using the local Bregman divergence
\begin{equation}
    \omega_t(f, g) = \dotp{\alpha}{\exp(f_t - f)} + \dotp{\beta}{\exp(g_t - g)},
\end{equation}
online Sinkhorn iterations rewrites as
\begin{equation}
    \nabla \omega_t(f_{t+1}, g_{t+1}) = \nabla \omega_t(f_t, g_t) + \eta_t \tilde \nabla \Ff(f_t, g_t),
\end{equation}
where 
\begin{equation}
    \tilde \nabla \Ff(f, g) = \Big(\hat \alpha_t \big(1 - \exp(f - T_\beta(g))\big), 
    \hat \beta_t \big(1 - \exp(g - T_\alpha(f))\big)\Big)
    \in \Mm^+(\Xx^2)
\end{equation}
is an unbiased estimate of $\nabla \Ff(f, g)$.

\section{An EM point of view}
The simultaneous Sinkhorn updates can be rewritten as
\begin{align}
    f_t, g_t = \argmax_{f, g} Q_t^\star((f, g), (f_t, g_t)) 
    &\triangleq \EE_{Y \sim \beta} \left[ \EE_{X \sim \alpha} \left[
     f(X) - e^{f(X) + g_t(Y) - C(X, Y)} \right] \right] \\
     &+
     \EE_{X \sim \alpha} \left[ \EE_{Y \sim \beta} \left[
     g(Y) - e^{f_t(X) + g(Y) - C(X, Y)} \right] \right].
\end{align}
This is similar to the EM algorithm: the first expectation is on data, the second on hidden random variables.
We now define the approximate functions

\begin{align}
    Q_t((f, g), (f_t, g_t)) &= \EE_{Y \sim \hat \beta_t} \left[ \EE_{X \sim \alpha} \left[
        f(X) - e^{f(X) + g_t(Y) - C(X, Y)} \right] \right] \\
        &+
        \EE_{X \sim \hat \alpha_t} \left[ \EE_{Y \sim \beta} \left[
        g(Y) - e^{f_t(X) + g(Y) - C(X, Y)} \right] \right] \\
        &= \EE_{X \sim \alpha} [f(X)] + \EE_{X \sim \alpha} \left[ \sum_{i=n_t}^{n_{t+1}} 
        b_i e^{f(X) + g_t(y_i) - C(X, y_i)} \right] \\
        &+ \EE_{Y \sim \beta} [g(Y)] + \EE_{Y \sim \beta} \left[ \sum_{i=n_t}^{n_{t+1}} 
        a_i e^{g(Y) + f_t(x_i) - C(x_i, Y)} \right]
\end{align}
Running the iterations
\begin{equation}
    f_t, g_t = \argmax_{f,g} Q_t((f, g), (f_t, g_t))
\end{equation}
amounts to set
\begin{equation}
    f_t(\cdot) = - \log \sum_{i=n_t}^{n_{t+1}} b_i e^{g_t(y_i) - C(\cdot, y_i)} \quad
    g_t(\cdot) = - \log \sum_{i=n_t}^{n_{t+1}} a_i e^{f_t(x_i) - C(x_i, \cdot)},
\end{equation}
which is the randomized Sinkhorn algorithm. Setting
\begin{equation}
    \bar Q_t = (1 - \eta_t) \bar Q_{t-1} + \eta_t Q_{t}
\end{equation}
and running the iterations
\begin{equation}
    f_t, g_t = \argmin_{f,g} \bar Q_t((f, g), (f_t, g_t))
\end{equation}
gives online Sinkhorn:
\begin{equation}
    f_t(\cdot) = - \log \sum_{i=1}^{n_{t+1}} e^{q_i - C(\cdot, y_i)} \quad
    g_t(\cdot) = - \log \sum_{i=1}^{n_{t+1}} e^{p_i - C(x_i, \cdot)},
\end{equation}
with the update rule on $q_i, p_i$ as : see paper.
Every function $Q_t$ is parametrized by $(p_i, q_i, x_i, y_i)_{i=(n_t,
 n_{t+1}]}$, and $\bar Q_t$ by $(p_i, q_i, x_i, y_i)_{i=(0, n_{t+1}]}$. Thus the
 parametrization of $f_t, g_t$ is encoded using an argmax trick, and we recover the structure of a stochastic expectation-maximization algorithm (less the probabilistic point of view).

 \section{Stochastic approximation}

 Online EM: in finite dimension: \fullcite{cappe_online_2009}

 Applications + better explanation: \fullcite{dupuy_online_2017}

 Random fixed point iterations: \fullcite{alber_stochastic_2012}

 Non-asymptotic rates for SGD + Polyak-Ruppert averaging: \fullcite{moulines_non-asymptotic_2011}

\subsection{The Robbins Monroe-Algorithm}
Overall, everything can be rewritten as looking for the zero of some function
\begin{equation}
    \text{Find}\:x^\star\:\text{such that}\:h(x) = 0,
\end{equation}
with access to an oracle 
$\hat h(x)$ s.t. $\EE[\hat h(x)] = h(x)$ for all $x \in \Xx$. Then the algorithm
\begin{equation}
    x_{n+1} = x_n - \eta_n h(x_n)
\end{equation}
gives a sequence converging to $x^\star$, provided that
\begin{equation}
    \sum_n \eta_n = \infty,\qquad \sum_n \eta_n^2 \leq \infty,
    \qquad h\text{ non decreasing}\qquad \EE[h(x_n)^2 | \Ff_{n-1} ] \leq \sigma^2
\end{equation}
When looking for $\min f(x)$, we can use $h(x) = \nabla f(x)$. When looking for a fixed point equation
\begin{equation}
    x = T x,
\end{equation}
we may use $h(x) = x - T(x)$, in which case the algorithm writes
\begin{equation}
    x_{n+1} = (1 - \eta_n) x_n + \eta_n S(x_n),
\end{equation}
where $\EE[S(x_n)] = x - T(x)$, which is our case. In a Hilbert space, assuming $T$ is contracting for the norm, i.e.
\begin{equation}
    \Vert Tx - Ty \Vert \leq \kappa \Vert x - y \Vert,
\end{equation}
it is easy to obtain convergence of $\EE[\Vert x_n  - x^\star \Vert^2]$ + rates on the mean-square convergence rate + almost sure convergence of the iterate \autocite{alber_stochastic_2012}.


\subsection{Proof: basic inequality}

Overall, all these references use at some point exhibits a sequence $(\delta_n)_n$ such that
\begin{equation}
    \delta_{n+1} \leq (1 - \eta_n) \delta_n + C \gamma_n
\end{equation}
with $\sum \eta_n = \infty$ and $\sum \gamma_n \leq \infty$. Typically $\gamma_n = \eta_n^2$.

E.g. from SGD, setting $\delta_n = \EE[|| \theta_n - \theta ||^2]$, we have, if objective is $L$-smooth and $\mu$-strongly convex:
\begin{equation}
    \delta_n \leq (1 - 2 \mu \gamma_n   + 2 L^2 \gamma_n^2) \delta_{n-1} + 2 \sigma^2 \gamma_n^2
\end{equation}


\paragraph{Problem.} We do not have access to such an equality:
\begin{itemize}
    \item The contraction of the Sinkhorn operator is for a non-Euclidean distance
    \item Therefore we need to increase the sampling size with time
\end{itemize}

What we have at hand, $e_t \triangleq \EE ||f_t - f^\star||_{var} + ||g_t - g^\star||_{var}$:
\begin{equation}\label{eq:Et_rec}
    0 \leq e_{t+1} \leq 
    (1 - \tilde \eta_t) e_t
    + \tilde \eta_t
    ({\Vert \epsilon_{\hat \beta_t} \Vert}_{\var} + 
    {\Vert \iota_{\hat \alpha_t} \Vert}_{\var}).
\end{equation}
with $\tilde \eta_t = \eta_t (1 - \kappa)$ and
\begin{equation}
    \epsilon_{\hat \beta}(\cdot) \triangleq
    f^\star - T_{\hat \beta}{g^\star} ,\qquad
    \iota_{\hat \alpha}(\cdot) \triangleq 
    g^\star - T_{\hat \alpha}{f^\star},
\end{equation}
With increasing batch-sizes, we may end up with
\begin{equation}
    e_{t+1} \leq (1 - \eta_t) e_t + C \eta_t w_t.           
\end{equation}

\section{Rates for online Sinkhorn}

We set $e_t \triangleq \Vert f^\star - f_t \Vert_{\var} + \Vert g^\star - g_t \Vert_{\var}$.
From Eq. (10) and Eq. (15) in the paper, there exists $A, A' >0$ such that
\begin{align}
    \delta_{t+ 1} = \EE e_{t+1} &\leq (1 - (1 - \kappa) \eta_t) \EE e_t + \eta_t 
    \frac{A}{\sqrt{n(t)}} \\
\end{align}
Note the $1 - \kappa$ that appears in the recursion 
(which breaks for $\varepsilon = 0$). We
set $\eta_t = \frac{S}{t^a}$, $n(t) = \lceil B t^{2b} \rceil$. We are left to study the recursion
\begin{equation}
    \delta_{t+1} \leq (1 - \frac{S(1 - \kappa)}{t^a}) + 
    \frac{A S}{\sqrt{B}{t^{a + b}}}
\end{equation}

Using the proof of \citep[Theorem 2]{moulines_non-asymptotic_2011}, we have,
provided that $0 \leq a < 1$ and $a+ b > 1$, for all $t > 0$,
% \begin{equation}
%     \delta_t \leq (\delta_0 + \frac{A S}{\sqrt{B}} \phi_{1 - (a + b)}(t))
%     \exp(- \frac{S(1 - \kappa)}{2} n^{1 - a})
%     + \frac{2 A S}{\sqrt{B}(1 - \kappa) n^a},
% \end{equation}
% where $\phi_c(t) = \frac{t^c - 1}{c}$ for all $c \neq 0$, $\phi_0(t) = \log t$.
% Note that $\phi_{1 - (a + b)} \leq \frac{1}{a + b - 1}$, so that
\begin{equation}
    \delta_t \leq (\delta_0 + \frac{A S}{(a + b - 1)\sqrt{B}})
    \exp(- \frac{S(1 - \kappa)}{2} t^{1 - a})
    + \frac{2 A S}{\sqrt{B}(1 - \kappa) t^a}.
\end{equation}
Let us now relate the iteration number $t$ to the number of seen sample $n(t)$. By definition
\begin{equation}
    n_t = \sum_{s=1}^t n(s) \leq B \sum_{s=1}^t s^{2b} + t \leq
     t + \frac{(t+1)^{2b + 1} - 1}{2b + 1}
     \leq \frac{2b + 1}{2b + 2} (2t)^{2b+1}.
\end{equation}
Therefore, when we have seen $n$ samples, the error we make is $\delta_t$, with
\begin{equation}
    t \geq {(n/2)}^{\frac{1}{2b + 1}}.
\end{equation}
Therefore
\begin{equation}
    \delta_n \triangleq \delta_{t+1} \leq 
    (\delta_0 + \frac{A S}{(a + b - 1)\sqrt{B}})
    \exp(- \frac{S(1 - \kappa)}{2} {(n /2)}^{\frac{1 - a}{2b+1}})
    + \frac{2 A S}{\sqrt{B}(1 - \kappa) {(n/2)}^{\frac{a}{2b+1}}}.
\end{equation}
We set $a = 1 - \iota$, $ b = 2 \iota$:
\begin{align}
    \delta_n \triangleq \delta_t &\leq 
    (\delta_0 + \frac{A S}{(a + b - 1)\sqrt{B}})
    \exp(- \frac{S(1 - \kappa)}{2} {(n/2)}^{\frac{\iota}{4\iota+1}})
    + \frac{2 A S}{\sqrt{B}(1 - \kappa) {(n/2)}^{\frac{1 - \iota}{4\iota+1}}} \\
    &\leq \Oo(n^{-\frac{1 - \iota}{4 \iota - 1}}).
\end{align}
Notice that $b$ and $a$ should be as close to $0$ as possible to reduce the bias term, while $a$
should be as close to $1$ and $b$ as close to $0$ as possible to reduce the variance
term (with respect to $n$). As the variance term dominates asymptotically, we take $a=1 - \iota$ and
$b = \iota$. Note that we can take $a = 1$ and $b = 1$, in which case the second part of Theorem 2 from \citet{moulines_non-asymptotic_2011} can still be applied and we get
\begin{equation}
    \delta_n = \Oo(\frac{1}{n^{\frac{1 - \kappa}{2}}})
\end{equation}

\section{Unbalanced algorithm}

Fixed point equation (KL divergence, or aprox from Thibault's paper)
\begin{equation}
    f^\star = \left( 1 + \frac{\varepsilon}{\rho} \right)^{-1} T_\beta ( g^\star), \qquad
    g^\star = \left( 1 + \frac{\varepsilon}{\rho} \right)^{-1} T_\alpha ( f^\star)
\end{equation}
In unbiased space, $\lambda \triangleq \left( 1 + \frac{\varepsilon}{\rho} \right)^{-1}$:
\begin{equation}
    u^\star = \exp(-f^\star) = \exp(-\lambda) \exp(-T_\beta ( g^\star)), \qquad
    v^\star = \exp(-g^\star) = \exp(-\lambda) \exp(-T_\alpha ( f^\star))
\end{equation}
Define
\begin{equation}
    T(u, v) \triangleq \left(\exp(-\lambda) \exp(-T_\beta (\log(v))),
                    \exp(-\lambda) \exp(-T_\alpha (-\log(u)) \right)
\end{equation}
fixed point operator. Online Sinkhorn reads
\begin{equation}
    x_n = (u_n, v_n) = (1 - \eta_n) x_{n-1} + \eta_n T_n(x_{n-1}),
\end{equation}
\begin{equation}
    T_n(u, v) \triangleq \left(\exp(-\lambda) \exp(-T_{\hat \beta_n} (\log(v))),
                    \exp(-\lambda) \exp(-T_{\hat \alpha_n} (-\log(u)) \right)
\end{equation}

\printbibliography

\end{document}
