%!TEX root = article.tex
\section{Introduction}

Optimal transport (OT) distances are fundamental in statistical learning, both as a tool for analyzing the convergence of various algorithms~\cite{canas2012learning,dalalyan2019user}, and as a data-dependent term for tasks as diverse as supervised learning~\cite{frogner2015learning}, unsupervised generative modeling~\cite{arjovsky2017wgan} or domain adaptation~\cite{courty2016optimal}.
%
OT lifts a given distance over data points living in space $\Xx$ into a distance
on the space $\Pp(\Xx)$ of probability distributions over this data space $\Xx$. We refer to the monograph~\cite{santambrogio2015optimal} for a detailed mathematical treatement.
%
This distance has many favorable geometrical properties. In particular it allows one to compare distributions having disjoint supports. 
% 
Computing OT distance is usually performed by sampling once from the input
distributions and solving a discrete linear program (LP), due to~\citet{Kantorovich42}. This approach is numerically costly and
statistically unefficient \cite{weed2019sharp}. The optimisation problem depends
on a fixed sampling of points from the data. It is therefore not adapted to
machine learning setting where data is resampled continuously (e.g. in GANs), or
accessed in an online manner. The goal of this paper is to develop an efficient
online method able to estimate OT distances between continuous distributions ---
we will use a stream of data to refine the optimal transport solution. For this, we will
adapt the celebrated Sinkhorn algorithm to an online setting.
  


%%%%
\paragraph{Regularized OT.}

To alleviate both the computational and statistical burdens of OT, it is common
to regularize the Kantorovich LP.
%
The most successful approach in this direction is to use an entropic barrier penalty. 
%
When dealing with discrete distributions, this yields a problem that can be solved
numerically using Sinkhorn-Knopp's matrix balancing
algorithm~\cite{Sinkhorn64,sinkhorn1967concerning}.
%
This approach was pushed forward for ML applications
by \citet{cuturi2013sinkhorn}. Sinkhorn distances are smooth and amenable to GPU computations, which makes them suitable as a loss function in model training.
%
Sinkhorn algorithm computes $\epsilon$-accurate approximations of OT in
$O(n^2/\epsilon^3)$ operations for a number $n$ of samples~\cite{altschuler2017near} (in
contrast to the $O(n^3)$ complexity for an exact solution). Moreover, these Sinkhorn distances does not suffers from the curse of dimensionality~\cite{2019-Genevay-aistats}, since the average error using $n$ random samples decays like $O(\epsilon^{-d/2}/\sqrt{n})$ in dimension $d$, in sharp contrast with the slow $O(1/n^{1/d})$ error decay of OT~\cite{weed2019sharp}. These Sinkhorn distances can furthermore be sharpened by entropic debiasing~\cite{2019-Feydy-aistats}.  

\paragraph{Handling streams of samples.} The Sinkhorn algorithm operates in two
distinct phases: draw samples from the distributions and evaluate a pairwise distance matrix in the
first phase ; balance this distance matrix using Sinkhorn-Knopp iterations in
the second phase. This two-step approach does not estimate the true regularized OT distance, and cannot handle samples provided as a stream, e.g. renewed
at each training iteration of an outer algorithm. A cheap fix is to use Sinkhorn over mini-batches
(see for instance~\citet{2018-Genevay-aistats} for an application to GANs). Yet
this introduces a strong estimation bias, especially in high dimension
---see~\citet{fatras2019learning} for a mathematical analysis. 

\paragraph{Continuous OT.} Extending OT computations to arbitrary distributions
(possibly having continuous densities) without relying on a fixed a priori
sampling is an emerging topic of interest. A special case is the semi-discrete
setting, where one of the two distributions is discrete. Without regularization,
over an Euclidean space, this can be solved efficiently using the computation of
Voronoi-like diagrams~\cite{merigot2011multiscale}. This idea can be extended to
entropic-regularized OT~\cite{cuturi2018semidual}, and can also be coupled with
stochastic optimization method~\cite{2016-genevay-nips} to tackle high
dimensional problems (see also~\citet{staib2017parallel} for an extension to Wasserstein barycenters). 

When dealing with arbitrary continuous densities, which are accessed through a
stream of random sample, the challenge is to approximate  the (continuous) dual
variables of the regularized Kantorovich LP using parametric or non-parametric functions. For application to
generative model fitting, one can use deep networks, which leads to alternative
formulation to Generative Adversarial Networks (GANs)~\cite{arjovsky2017wgan}
(see also~\citet{seguy2018large} for an extension to the estimation of
transportation maps). There is however no theoretical guarantees for this type
of dual approximations, due to the non-convexity of the resulting optimization
problem. The only mathematically rigorous approach in this direction is to use
reproducing Hilbert space representations of
potentials~\cite{2016-genevay-nips}. However, this approach relies on generic functional
approximations, that results in very slow convergence.


\paragraph{Contribution.}

Our paper proposes a new take on continuous OT estimation. We construct a
non-parametric representation of the potentials by extending the discrete Sinkhorn algorithm to the continuous setting.


% In this paper, we show how mingling together the sampling and the optimization phases can be beneficial
% to quickly estimate OT distances for ML problems.  
% %
% Our approach relies on three observations. First,
% Sinkhorn iterations can be rewritten as a block convex mirror descent on the
% space of positive distributions. This formulation is valid in the discrete and
% continuous setting. Second, we can modify these iterations to rely on
% realizations $\hat \alpha_t$, $\hat \beta_t$ of the input distributions $\alpha$
% and $\beta$, renewed at each iteration $t$. Finally, we can represent the
% iterates produced by such approximations in a space of mixtures of simple
% functions. Those iterates are a simple transformation of the potentials in the
% Sinkhorn optimization problem.
% %
% This corresponds to the following contributions:
% \begin{itemize}
%     \item We introduce a new \textit{online Sinkhorn} algorithm. It produces a sequence of
%     estimates $(\hat w_t)_t \in \RR$ and of transportation plans $\hat \pi_t \in
%     \Pp(\Xx \times \Xx)$ \todo{need to introduce this}, using two stream of renewed samples $\hat \alpha_t =
%     \sum_{i=1}^n \delta_{x^t_i}$, $\hat \beta_t = \sum_{i=1}^n \delta_{y^t_i}$,
%     where $x^t_i$ and $y^t_i$ are sampled from $\alpha$ and $\beta$.
%     \item We show that those estimations are consistent, in the sense that $\hat
%     w_t \to \Ww_{C,\varepsilon}(\alpha, \beta)$, and $\hat \pi_t \to \pi^\star$ \todo{weak convergence?}.
%     \item We empirically demonstrate that our algorithm permits a faster
%     estimation of optimal transportation distances for discrete distributions,
%     and a convincing estimation of OT distances between \textit{continous} distributions.
% \end{itemize}

% % We show how to estimate the true regularized optimal transport distance through repeated experiments. 